\documentclass[10pt]{article}
% Эта строка — комментарий, она не будет показана в выходном файле
\usepackage{ucs}
\usepackage[a4paper, total={6in, 10in}]{geometry}
\usepackage{listings}
\usepackage[utf8x]{inputenc } % Включаем поддержку UTF8
\title{Essential Skills. Lab 1}
\date{02 September 2016}
\author{Ali Abdulmadzhidov}

\begin{document}
\renewcommand*\rmdefault{cmss}
  \maketitle
  \section{Step by step DokuWiki installation and setup     \newline}

  \begin{enumerate}
    \item Checking apache2 installed
    \begin{verbatim}
    service apache2 status
    \end{verbatim}
    \item Installing DokuWiki
    \begin{verbatim}
    sudo apt install dokuwiki
    \end{verbatim}
    \item Set password for user admin in wiki while installing
    \item Realizing that we haven't enabled php in apache2 config. 
    \begin{verbatim}
    nano /etc/apache2/apache2.conf
    LoadModule php7_module modules/libphp7.so 
    \end{verbatim}
    \item Realizing that we haven't got libapache2-mod-php and apache2 won't start without it. 
    \begin{verbatim}
    sudo apt install libapache2-mod-php 
    \end{verbatim}
    \item Restarting apache2. 
    \begin{verbatim}
    sudo service apache2 restart
    \end{verbatim}
    \item Go to http://localhost/dokuwiki/doku.php and add our first page about Federative Wiki
  \end{enumerate}

  \section{XML entities creating    \newline}
  \begin{enumerate}
    \item XML + DDT
    \begin{verbatim}
        <!DOCTYPE root
        [
        <!ELEMENT root (animal)>
        <!ELEMENT animal (name,skin,voice)>
        <!ELEMENT name (#PCDATA)>
        <!ELEMENT skin (#PCDATA)>
        <!ELEMENT voice (#PCDATA)>
        <!ENTITY  skin "have skin">
        <!ENTITY  fur "have fur">
        <!ENTITY  feathers "have feathers">
        <!ENTITY  bark "bark">
        <!ENTITY  mur "mur">
        <!ENTITY  bee "bee">
        <!ENTITY  brr "brr">
        <!ENTITY  mooo "mooo">
        <!ENTITY  oink "oink">
        <!ENTITY  whistle "whistle">
        <!ENTITY  no_voice "no_voice">
        ]>
        <root>
<animal>
            <name>cow</name>
            <skin>&skin;</skin>
            <voice>&mooo;</voice>
        </animal>
        <animal>
            <name>sheep</name>
            <skin>&fur;</skin>
            <voice>&bee;</voice>
        </animal>
        <animal>
            <name>horse</name>
            <skin>&skin;</skin>
            <voice>&brr;</voice>
        </animal>
        <animal>
            <name>pig</name>
            <skin>&skin;</skin>
            <voice>&oink;</voice>
        </animal>
        <animal>
            <name>mockingbird</name>
            <skin>&feathers;</skin>
            <voice>&whistle;</voice>
        </animal>
        <animal>
            <name>eal</name>
            <skin>&feathers;</skin>
            <voice>&whistle;</voice>
        </animal>
</root>
    \end{verbatim}
    \item XS SCHEME
    \begin{verbatim}
    <?xml version="1.0" encoding="utf-8"?>
    <xs:schema xmlns:xs="http://www.w3.org/2001/XMLSchema">
    <xs:element name="root">
	<xs:complexType>
          <xs:sequence>
      	<xs:element name="animal" maxOccurs="unbounded" >
        <xs:complexType>
          <xs:sequence>
            <xs:element name="name" type="xs:string"/>
            <xs:element name="voice" type="xs:string"/>
            <xs:element name="skin" type="xs:string"/>
          </xs:sequence>
        </xs:complexType>
      </xs:element>
          </xs:sequence>
        </xs:complexType>
      </xs:element>
    </xs:schema>
    \end{verbatim}
  \end{enumerate}
\section*{XSLT and CSS \\}
\begin{enumerate}
\item XSLT
\begin{verbatim}
<?xml version='1.0' encoding='UTF-8'?>
<xsl:stylesheet version='1.0' xmlns:xsl='http://www.w3.org/1999/XSL/Transform'>
<xsl:template match='/'>
  <html>
  <body>
  <h2>Animals</h2>
    <table border='1'>
      <tr>
        <th style='text-align:left'>Name</th>
        <th style='text-align:left'>Skin</th>
        <th style='text-align:left'>Voice</th>
      </tr>
      <xsl:for-each select='root/animal'>
      <tr>
        <td class='name-vl'><xsl:value-of select='name'/></td>
        <td class='skin-vl'><xsl:value-of select='skin'/></td>
        <td class='voice-vl'><xsl:value-of select='voice'/></td>
      </tr>
      </xsl:for-each>
    </table>
  </body>
  </html>
</xsl:template>
</xsl:stylesheet>
\end{verbatim}

\item CSS
\begin{verbatim}
td {
    background: #eeccbb;
}
th {
    background: #ffff00;  
}
.name-vl {
    color: #0f0;
}
.name-vl:hover {
    color: #00f;
}
.voice-vl {
    color: #00f;
}
.voice-vl:hover {
    color: #0f0;
}
.skin-vl {
    color: #f00;
}
.skin-vl:hover {
    color: #f0f;
}
\end{verbatim}
\end{enumerate} 
\end{document}
