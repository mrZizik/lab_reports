\documentclass[10pt]{article}
\usepackage{ucs}
\usepackage[a4paper, total={6in, 10in}]{geometry}
\usepackage{listings}
\usepackage[utf8x]{inputenc } % Включаем поддержку UTF8
\usepackage{graphicx}
\usepackage{natbib}
\title{Essential Skills. Lab 4}
\date{04 October 2016}
\author{Ali Abdulmadzhidov}

\begin{document}
\renewcommand*\rmdefault{cmss}
\maketitle
\section{History of git commands}
\begin{enumerate}
    \item git init - initiliazies new git repo on your local machine.
    \item git status - shows current state of repo
    \begin{verbatim}
        >git status
        # On branch master
        # Changes not staged for commit:
        #   (use "git add <file>..." to update what will be committed)
        #   (use "git checkout -- <file>..." to discard changes in working directory)
        #
        #   modified:   main.tex
        #
        no changes added to commit (use "git add" and/or "git commit -a")
    \end{verbatim}
    \item git clone - Clones remote repo to local one, where you can modify it.
    \begin{verbatim}
        git clone git@st9.os3.su:/home/git/project.git
    \end{verbatim}
    \item git add - adds files from working dir to staging area for further commit
    \begin{verbatim}
        git add some_file.php
    \end{verbatim}
    \item git checkout - undos git add, and frees index.
    \item git commit - commits and fixes all from index on local repo
    \begin{verbatim}
        git commit -m "commit_message"
    \end{verbatim}
    \item git push - pushes our new commits to remote repo.
    \begin{verbatim}
        git commit -m "commit_message"
    \end{verbatim}
\end{enumerate}

\section{Link with credintials}
    \begin{verbatim}
        git clone ssh://git@st9.os3.su:1022/home/git/project.git
        password: git
    \end{verbatim}


\section{Commands to compile}
    \begin{verbatim}
        make
        make clean # to remove all temp files.
    \end{verbatim}



\section{Makefile}
    \begin{verbatim}
        filename=main
        pdf: 
            pdflatex ${filename}
        clean:
            rm -f *.ps *.aux *.toc *.bbl *.blg *.log
    \end{verbatim}

\end{document}