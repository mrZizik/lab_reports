\documentclass[10pt]{article}
\usepackage{ucs}
\usepackage[a4paper, total={6in, 10in}]{geometry}
\usepackage{listings}
\usepackage[utf8x]{inputenc } % Включаем поддержку UTF8
\usepackage{graphicx}
\usepackage{natbib}
\title{Essential Skills. Lab 4}
\date{04 October 2016}
\author{Ali Abdulmadzhidov}

\begin{document}
\renewcommand*\rmdefault{cmss}
\maketitle

\section{Git server creating}
\begin{enumerate}
	\item Adding user and logging in as him
	\begin{verbatim}
		> sudo adduser git
		> su git
	\end{verbatim}
	\item Initiating git bare repository from git user
	\begin{verbatim}
		> mkdir ~/project && cd ~/project
		> git --bare init
	\end{verbatim}
	\item Exiting from git user
	\begin{enumerate}
		> exit
	\end{enumerate}
	\item Go to /etc/passwd and block shell for that user
	\begin{verbatim}
		> sudo nano /etc/passwd
		...
		git:x:1002:1002:git,1,123321,12321,a:/home/git:/usr/bin/git-shell
		...
	\end{verbatim}
\end{enumerate}

\section{History of git commands}
\begin{enumerate}
    \item git init - initiliazies new git repo on your local machine.
    \item git stasdasdasd
        no cashanges added to commit (use "git add" and/or "git commit -a")
    \end{verbatim}
    \item git clone - Clones remote repo to local one, where you can modify it.
    \begin{verbatim}
        git clone git@st9.os3.su:/home/git/project.git
    \end{verbatim}
    \item git add - adds files from working dir to staging area for further commit
    \begin{verbatim}
        git add some_file.php
    \end{verbatim}
    \item git checkout - undos git add, and frees index.
    \item git commit - commits and fixes all from index on local repo
    \begin{verbatim}
        git commit -m "commit_message"
    \end{verbatim}
    \item git push - pushes our new commits to remote repo.
    \begin{verbatim}
        git commit -m "commit_message"
    \end{verbatim}
\end{enumerate}

\section{Link with credintials}
    \begin{verbatim}
        git clone ssh://git@st9.os3.su:1022/home/git/project.git
        password: git
    \end{verbatim}


\section{Commands to compile}
    \begin{verbatim}
        make
        make clean # to remove all temp files.
    \end{verbatim}



\section{Makefile}
    \begin{verbatim}
        filename=main
        pdf: 
            pdflatex ${filename}
        clean:
            rm -f *.ps *.aux *.toc *.bbl *.blg *.log
    \end{verbatim}


Main latex file
\begin{verbatim}
\documentclass[10pt]{article}
% Эта строка — комментарий, она не будет показана в выходном файле
\usepackage{ucs}
\usepackage[a4paper, total={6in, 10in}]{geometry}
\usepackage{listings}
\usepackage[utf8x]{inputenc } % Включаем поддержку UTF8
\usepackage{graphicx}
\title{Essential Skills. Lab 4}
\date{25 September 2016}
\author{Ali Abdulmadzhidov, Oleg Ilin, Timur Samigullin}

\begin{document}
\renewcommand*\rmdefault{cmss}
\maketitle

\tableofcontents
\newpage

\documentclass[column]{report}
\usepackage{graphicx}
\usepackage[utf8]{inputenc}
\usepackage{hyperref}
% \usepackage[
%   margin=1.5cm,
%   includefoot,
%   footskip=30pt,
% ]{geometry}
\usepackage{lipsum}
\begin{document}

\graphicspath{ {./images/} }

\title{\includegraphics{logo}\\Research Project 1}
\author{Emil Sharifullin, Ali Abdulmadzhidov}
\date{\today}

\maketitle
\tableofcontents
\listoffigures
\listoftables
\newpage
\chapter{Introduction}
Nowadays, mobile devices are very popular and their "expansion" is growing. In addition to usual smartphones, we have IoT that also uses same mobile OSs (iWatch, Samsung Gear). As a result, users trust very sensitive data to them, gave access to their bank accounts e.t.c. This fact makes mobile devices sweet target for developers of malware. According to McAfee mobile threat report, there're 1000 - 6000 virus detected per hour in various countries.
   
Currently not all applications can be verified and compared via traditional ways like a hash sums or signs. Mobile applications can be changed or repacked with some malicious code before installation. Target of our work is to determine malicious behaviour of repacked applications based on their resources consumption. In our work we plan to test the usage of different resources such as CPU, battery and so on to determine that malicious activity.
\chapter{Contribution}
\section{Table}

\begin{table}[!th]
\begin{center}
\begin{tabular}{|c|c|}
\hline
\# & Test \\ \hline \hline
1 & one \\ \hline
\end{tabular}
\caption{Table}
\label{ex:table}
\end{center}
\end{table}

\section{Figure}
\begin{figure}[h]
\includegraphics[width=5cm]{logo}
\caption{figure example}
\label{ex:figure}
\end{figure}


\section{Math}
\subsection{With numbering}
This sunsection contains math with numbering
\begin{equation}
x=\frac{x+z/2}{y^2+1}
\label{ex:equation}
\end{equation}

\subsection{Without numbering}
This sunsection contains math without numbering $x=\frac{x+z/2}{y^2+1}$

\section{Second chapter}
\subsection{Lists}
\subsubsection{Bullets}

\begin{itemize}
	\item First
	\item Second
\end{itemize}

\subsection{Enumerated list}
\begin{enumerate}
	\item First item
	\item Second item
\end{enumerate}

\subsection{Fish text}
\lipsum
\chapter{Conclusion}
\section{First}
\subsection{Fish text}
\lipsum

\subsection{Lists}
\subsubsection{Bullets}
\begin{itemize}
	\item First
	\item Second
\end{itemize}

\subsection{Enumerated list}
\begin{enumerate}
	\item First item
	\item Second item
\end{enumerate}

\bibliography{main}
\bibliographystyle{plain}
\end{document}

\newpage
\documentclass[column]{report}
\usepackage{graphicx}
\usepackage[utf8]{inputenc}
\usepackage{hyperref}
% \usepackage[
%   margin=1.5cm,
%   includefoot,
%   footskip=30pt,
% ]{geometry}
\usepackage{lipsum}
\begin{document}

\graphicspath{ {./images/} }

\title{\includegraphics{logo}\\Research Project 1}
\author{Emil Sharifullin, Ali Abdulmadzhidov}
\date{\today}

\maketitle
\tableofcontents
\listoffigures
\listoftables
\newpage
\chapter{Introduction}
Nowadays, mobile devices are very popular and their "expansion" is growing. In addition to usual smartphones, we have IoT that also uses same mobile OSs (iWatch, Samsung Gear). As a result, users trust very sensitive data to them, gave access to their bank accounts e.t.c. This fact makes mobile devices sweet target for developers of malware. According to McAfee mobile threat report, there're 1000 - 6000 virus detected per hour in various countries.
   
Currently not all applications can be verified and compared via traditional ways like a hash sums or signs. Mobile applications can be changed or repacked with some malicious code before installation. Target of our work is to determine malicious behaviour of repacked applications based on their resources consumption. In our work we plan to test the usage of different resources such as CPU, battery and so on to determine that malicious activity.
\chapter{Contribution}
\section{Table}

\begin{table}[!th]
\begin{center}
\begin{tabular}{|c|c|}
\hline
\# & Test \\ \hline \hline
1 & one \\ \hline
\end{tabular}
\caption{Table}
\label{ex:table}
\end{center}
\end{table}

\section{Figure}
\begin{figure}[h]
\includegraphics[width=5cm]{logo}
\caption{figure example}
\label{ex:figure}
\end{figure}


\section{Math}
\subsection{With numbering}
This sunsection contains math with numbering
\begin{equation}
x=\frac{x+z/2}{y^2+1}
\label{ex:equation}
\end{equation}

\subsection{Without numbering}
This sunsection contains math without numbering $x=\frac{x+z/2}{y^2+1}$

\section{Second chapter}
\subsection{Lists}
\subsubsection{Bullets}

\begin{itemize}
	\item First
	\item Second
\end{itemize}

\subsection{Enumerated list}
\begin{enumerate}
	\item First item
	\item Second item
\end{enumerate}

\subsection{Fish text}
\lipsum
\chapter{Conclusion}
\section{First}
\subsection{Fish text}
\lipsum

\subsection{Lists}
\subsubsection{Bullets}
\begin{itemize}
	\item First
	\item Second
\end{itemize}

\subsection{Enumerated list}
\begin{enumerate}
	\item First item
	\item Second item
\end{enumerate}

\bibliography{main}
\bibliographystyle{plain}
\end{document}

\newpage
\documentclass[column]{report}
\usepackage{graphicx}
\usepackage[utf8]{inputenc}
\usepackage{hyperref}
% \usepackage[
%   margin=1.5cm,
%   includefoot,
%   footskip=30pt,
% ]{geometry}
\usepackage{lipsum}
\begin{document}

\graphicspath{ {./images/} }

\title{\includegraphics{logo}\\Research Project 1}
\author{Emil Sharifullin, Ali Abdulmadzhidov}
\date{\today}

\maketitle
\tableofcontents
\listoffigures
\listoftables
\newpage
\chapter{Introduction}
Nowadays, mobile devices are very popular and their "expansion" is growing. In addition to usual smartphones, we have IoT that also uses same mobile OSs (iWatch, Samsung Gear). As a result, users trust very sensitive data to them, gave access to their bank accounts e.t.c. This fact makes mobile devices sweet target for developers of malware. According to McAfee mobile threat report, there're 1000 - 6000 virus detected per hour in various countries.
   
Currently not all applications can be verified and compared via traditional ways like a hash sums or signs. Mobile applications can be changed or repacked with some malicious code before installation. Target of our work is to determine malicious behaviour of repacked applications based on their resources consumption. In our work we plan to test the usage of different resources such as CPU, battery and so on to determine that malicious activity.
\chapter{Contribution}
\section{Table}

\begin{table}[!th]
\begin{center}
\begin{tabular}{|c|c|}
\hline
\# & Test \\ \hline \hline
1 & one \\ \hline
\end{tabular}
\caption{Table}
\label{ex:table}
\end{center}
\end{table}

\section{Figure}
\begin{figure}[h]
\includegraphics[width=5cm]{logo}
\caption{figure example}
\label{ex:figure}
\end{figure}


\section{Math}
\subsection{With numbering}
This sunsection contains math with numbering
\begin{equation}
x=\frac{x+z/2}{y^2+1}
\label{ex:equation}
\end{equation}

\subsection{Without numbering}
This sunsection contains math without numbering $x=\frac{x+z/2}{y^2+1}$

\section{Second chapter}
\subsection{Lists}
\subsubsection{Bullets}

\begin{itemize}
	\item First
	\item Second
\end{itemize}

\subsection{Enumerated list}
\begin{enumerate}
	\item First item
	\item Second item
\end{enumerate}

\subsection{Fish text}
\lipsum
\chapter{Conclusion}
\section{First}
\subsection{Fish text}
\lipsum

\subsection{Lists}
\subsubsection{Bullets}
\begin{itemize}
	\item First
	\item Second
\end{itemize}

\subsection{Enumerated list}
\begin{enumerate}
	\item First item
	\item Second item
\end{enumerate}

\bibliography{main}
\bibliographystyle{plain}
\end{document}

\end{document}
\end{verbatim}

My latex file ali/main.tex
\begin{verbatim}
\section{Starting with git and github}
\textbf{Git} \cite{git} is version control system created by \textit{Linus} Torvalds in 2005 for development of linux kernel. For nowadays it is the most popular version system control. 
Github \cite{github} (figure \ref{logo}) is webbased git repository storage.
\begin{figure}[ht!]
\centering
\includegraphics[width=90mm]{ali/github-logo.jpg}
\caption{Github logo \label{logo}}
\end{figure}
\begin{enumerate}
    \item Firstly you need to install it
    \begin{verbatim}
        apt install git
    end{verbatim}
    \item Setup your credintials
    \begin{verbatim}
        git config --global user.name "Ali Abdulmadzhidov"
        git config --global user.email "a.abdulmadzhidov@innopolis.ru"
    end{verbatim}
    \item Now you can init new...
    \begin{verbatim}
        mkdir project
        git init
    end{verbatim}
    \item Now you can init new...
    \begin{verbatim}
        mkdir project
        git init
    end{verbatim}
    \item ...or clone exisitng one \cite{swipecards}
    \begin{verbatim}
        git clone https://github.com/Diolor/Swipecards.git 
    end{verbatim}
    \item You can look to repo's status with
    \begin{verbatim}
        git status
    end{verbatim}
    \item And watch all modifications in files with
    \begin{verbatim}
        git diff
    end{verbatim}
    \item You can add edited files to index with
    \begin{verbatim}
        git add <filename>
    end{verbatim}
    \item and commit them
    \begin{verbatim}
        git commit -m <message>
    end{verbatim}
    \item Push modifications to remote repo
    \begin{verbatim}
        git push origin <branch>
    end{verbatim}
    \item Pull modifications from remote repo
    \begin{verbatim}
        git pull
    end{verbatim}
    
\end{enumerate}

\medskip
 
\begin{thebibliography}{9}
\bibitem{git} 
https://ru.wikipedia.org/wiki/Git
 
\bibitem{github} 
https://github.com/
 
\bibitem{swipecards} 
https://github.com/Diolor/Swipecards
\end{thebibliography}
\end{verbatim}

\end{document}
