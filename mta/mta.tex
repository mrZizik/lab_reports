\documentclass[10pt]{article}
% Эта строка — комментарий, она не будет показана в выходном файле
\usepackage{ucs}
\usepackage[a4paper, total={6in, 10in}]{geometry}
\usepackage{listings}
\usepackage[utf8x]{inputenc } % Включаем поддержку UTF8
\title{CIA MTA. Lab 6}
\date{07 September 2016}
\author{Ali Abdulmadzhidov}

\begin{document}
\renewcommand*\rmdefault{cmss}
  \maketitle
  \section{Step by step Sendmail installation and setup     \newline}
    \begin{enumerate}
        \item Installing sendmail from repositories.
        \begin{verbatim}sudo apt install sendmail\end{verbatim}
        \item Added line to sendmail.mc to turn on sendmail on external interfaces.
        \begin{verbatim}DAEMON_OPTIONS(`Port=smtp,Addr=188.130.155.42, Name=MTA')dnl\end{verbatim}
        \item Adding new user to system, for mail purpose with shell false.
        \begin{verbatim}sudo useradd -s /bin/false a.abdulmadzhidov\end{verbatim}
        \item Adding access rules to files (used standart one). Added aliases for mail for my name to be sended to username mailbox. Also added alias for postmaster to root.
            ali: a.abdulmadzhidov
            postmaster: root
        \item Creating config, aliases, access db files.
        \begin{verbatim}make\end{verbatim}
        \begin{verbatim}cd /etc/mail\end{verbatim}
        \begin{verbatim}makemap hash access < access\end{verbatim}
        \begin{verbatim}makemap hash aliases < aliases\end{verbatim}
        \item Starting sendmail
        \begin{verbatim}sudo service sendmail start\end{verbatim}
        \item Checking it's running
        \begin{verbatim}sudo service sendmail status\end{verbatim}
    \end{enumerate} 
    \section*{Answers}
    \begin{enumerate}
       \item 
            \begin{enumerate}
                \item What are you need to add to you DNS zone?
                MX record for mail.st9.os3.su
                GLUE A record for record above
                PTR record from 42.155.130.188.in-addr.arpa. to mail.st9.os3.su
                \item What extra information should be modified in PTR records for your IP, please provide this information to Azat.?
                He should add PTR record for our ip to mail.st9.os3.su
            \end{enumerate} 
        \item 
            \begin{enumerate}
                \item Add a local account with your name.surename@stX.os3.su on your experimental machine and make sure that the MTA can deliver mail to it. Show the required configuration.
                \\
                Done in installation. We need to add user to system by useradd command, and add him to aliases file pointing usermname for mail and username in system who'll get mail for that user.
                \item Log
                \begin{verbatim}
From mrfreezzzer@mail.ru  Sun Sep 18 22:32:47 2016
Return-Path: <mrfreezzzer@mail.ru>
Received: from f330.i.mail.ru (f330.i.mail.ru [217.69.140.226])
    by mail.st9.os3.su (8.15.2/8.15.2) with ESMTP id u8IJWlqw022764
    for <ali@mail.st9.os3.su>; Sun, 18 Sep 2016 22:32:47 +0300
DKIM-Signature: v=1; a=rsa-sha256; q=dns/txt; c=relaxed/relaxed; d=mail.ru; s=mail2;
    h=Content-Type:Message-ID:Reply-To:Date:MIME-Version:Subject:To:From; bh=fZ/TjBBqITo3TadkefY4XHR+QFq3hEXRxXdFi+zxIHs=;
    b=NEoH02bhDCvx1/U5CFyAOlRUPqPTctvMxpGaQ2kYq94yqrMMNGpgOou7tAt8q6zfeUifLj1Lo6s95AJzkx8GgCdCxVgj/h37TNNXhu8EI+ynUIzne8LBBa6ECYKkwqxWC2IebIqsEpRhS5YqZH9Am/joHGxQRIuqbrf6c/c/qCk=;
Received: from [188.130.155.154] (ident=mail)
    by f330.i.mail.ru with local (envelope-from <mrfreezzzer@mail.ru>)
    id 1blhpW-0001Br-EC
    for ali@mail.st9.os3.su; Sun, 18 Sep 2016 22:32:46 +0300
Received: from [188.130.155.154] by e.mail.ru with HTTP;
    Sun, 18 Sep 2016 22:32:46 +0300
From: =?UTF-8?B?0JDQu9C4INCQ0LHQtNGD0LvQvNCw0LbQuNC00L7Qsg==?= <mrfreezzzer@mail.ru>
To: =?UTF-8?B?YWxp?= <ali@mail.st9.os3.su>
Subject: =?UTF-8?B?Z2hqZ2hq?=
MIME-Version: 1.0
X-Mailer: Mail.Ru Mailer 1.0
X-Originating-IP: [188.130.155.154]
Date: Sun, 18 Sep 2016 22:32:46 +0300
Reply-To: =?UTF-8?B?0JDQu9C4INCQ0LHQtNGD0LvQvNCw0LbQuNC00L7Qsg==?= <mrfreezzzer@mail.ru>
X-Priority: 3 (Normal)
Message-ID: <1474227166.17218210@f330.i.mail.ru>
Content-Type: multipart/alternative;
    boundary="--ALT--cL7RylSXAt3xuXh9IxOtpViiLeuCqAb71474227166"
Authentication-Results: f330.i.mail.ru; auth=pass smtp.auth=mrfreezzzer@mail.ru smtp.mailfrom=mrfreezzzer@mail.ru
X-95568C8E: 26815A15162540332840C59422DF1D5F199E47233F92776887959EFD9D048BA7C7B69D01972835153FC638AC42481FF8202895C63B50EB1067A1F204FDFC066C056B84F00A4844CEA8B0A4CD5757725850096BB181122863DC486A557FBD259A02A25C8A631CD0830FAD040E62B240F983FAB1837865CABA56D3983D7DC64F3B37AB95548734FFC173D42130AE1FBFDDE80FFA8285C16F69E80962FF9BA714797F46F790C81E7CE89706376B72A28421D76FB7D251B0F1EB09043B40D620717B56FDDA3F302AC58A0B3E7816C5E096C79ADF67231DAE8492AC54829705B7485015FEE5D97E0C5D7E410B9203F5845215D29A4D576C8C33EFCC02425DCD1736530A9E56D228FD3B34F7840D99C63BE10C494F1A4A73623B6BAA99C1767DC8DF702C7B639B9D57591C6A7B5F82DAE1CA9C1687EA4C5593F41C619F1BA6E5ED2B42BC42BABAB7E101F77E71BF67B0FE3FA2248190AD017C3BB6F4E2C5D00449FA05D9F3D6D224100BE43373D8D537493728E883161B769C70BB
X-Mailru-Sender: F602C57692161D0AE7BEC8AF795BFC16F517E54B8E90ABBF77F13EDC6E3CFDDC8457918EA0E798C0F0D203FE66146AB1
X-Mras: OK
X-Spam: undefined


----ALT--cL7RylSXAt3xuXh9IxOtpViiLeuCqAb71474227166
Content-Type: text/plain; charset=utf-8
Content-Transfer-Encoding: base64
YmJramhqaAoKCtChINGD0LLQsNC20LXQvdC40LXQvCwK0JDQu9C4INCQ0LHQtNGD0LvQvNCw0LTQ
ttC40LTQvtCyCmFiZHVsbWFkamlkb3ZAbGl2ZS5ydQ==

    ----ALT--cL7RylSXAt3xuXh9IxOtpViiLeuCqAb71474227166
    Content-Type: text/html; charset=utf-8
    Content-Transfer-Encoding: base64

    CjxIVE1MPjxCT0RZPmJia2poamg8YnI+PGJyPjxicj7QoSDRg9Cy0LDQttC10L3QuNC10LwsPGJy
    PtCQ0LvQuCDQkNCx0LTRg9C70LzQsNC00LbQuNC00L7Qsjxicj5hYmR1bG1hZGppZG92QGxpdmUu
    cnU8L0JPRFk+PC9IVE1MPgo=

    ----ALT--cL7RylSXAt3xuXh9IxOtpViiLeuCqAb71474227166--
                \end{verbatim}


                \item Log telnet
                \begin{verbatim}
telnet 127.0.0.1 25
Trying 127.0.0.1...
Connected to 127.0.0.1.
Escape character is '^]'.
220 mail.st9.os3.su ESMTP Sendmail 8.15.2/8.15.2; Sun, 18 Sep 2016 23:43:30 +0300
ehlo mail.st9.os3.su
250-mail.st9.os3.su Hello localhost [127.0.0.1], pleased to meet you
250-ENHANCEDSTATUSCODES
250-PIPELINING
250-EXPN
250-VERB
250-8BITMIME
250-SIZE
250-DSN
250-ETRN
250-AUTH DIGEST-MD5 CRAM-MD5
250-DELIVERBY
250 HELP
mail from: st9.os3.su
553 5.5.4 st9.os3.su... Domain name required for sender address st9.os3.su
mail from: ali@mail.st9.os3.su
250 2.1.0 ali@mail.st9.os3.su... Sender ok
rcpt to: mrfreezzzer@mail.ru
250 2.1.5 mrfreezzzer@mail.ru... Recipient ok
data
354 Enter mail, end with "." on a line by itself
text
.
250 2.0.0 u8IKhUSX023153 Message accepted for delivery
                \end{verbatim}
            \end{enumerate} 
    \end{enumerate} 

    \section{Mail Backup \\}
    \subsection{'First, describe you have done on your own server to create two backup MTAs for
your domain.'\\}
        We need to add two MX resource records to our zone, with lower priority.
        \begin{verbatim}
            mail            IN      MX      10       mail.st15.os3.su.
            mail            IN      MX      5        mail.st12.os3.su.
        \end{verbatim}
        10 and 5 in records are priority. Greater number, lower priority. In our zone, mail would go to mail.st12 and if it fails there too, it'll go to mail.st15


    \subsection{'Setting up our sendmail server as backup for other two\\}
        \begin{enumerate}
            \item Add to features to sendmail.mc. access\_db to allow relay not our mails to needed mail servers, and mailertable that shows which emails on what server should we send.
            \begin{verbatim}
            FEATURE(access_db, `hash -o -T /etc/mail/access')
            FEATURE(mailertable, `hash -o -T /etc/mail/mailertable')
            \end{verbatim}
            \item Add rules to route mails to /etc/mail/mailertable
            \begin{verbatim}
            mail.st12.os3.su    smtp:mail.st12.os3.su
            st12.os3.su    smtp:mail.st12.os3.su
            mail.st15.os3.su    smtp:mail.st15.os3.su
            st15.os3.su    smtp:mail.st15.os3.su
            \end{verbatim}
            \item Add rules to access relay for mail server that we are backuping to /etc/mail/access
            \begin{verbatim}
            TO:mail.st12.os3.su RELAY
            TO:st12.os3.su  RELAY
            TO:st15.os3.su  RELAY
            TO:mail.st15.os3.su RELAY
            \end{verbatim}
            \item To compile config files and create hash db files from mailertable and access files we need to call make all command in /etc/mail folder
            \item reloading config files in sendmail
            \begin{verbatim}
            service sendmail reload
            \end{verbatim}
        \end{enumerate}



        \subsection{Log of backup and getting my mail}
        We can see that sudent15@mail.st15.os3.su was sending it to my root@mail.st9.os3.su it went through mail.st12.os3.su and when i turned on my sendmail it recieved to me.
        \begin{verbatim}
        From sudent15@st15.os3.su  Mon Sep 19 13:26:49 2016
        Return-Path: <sudent15@st15.os3.su>
        Received: from st12.os3.su (mail.st12.os3.su [188.130.155.45])
            by mail.st9.os3.su (8.15.2/8.15.2) with ESMTP id u8JAQmYE029325
            for <root@mail.st9.os3.su>; Mon, 19 Sep 2016 13:26:49 +0300
        Received: from mail.st15.os3.su ([188.130.155.48] helo=st15.os3.su)
            by st12.os3.su with esmtp (Exim 4.87_167-ff5929e)
            (envelope-from <sudent15@st15.os3.su>)
            id 1bkqO9-0006Bs-AL
            for root@mail.st9.os3.su; Fri, 16 Sep 2016 13:28:57 +0300
        Received: by st15.os3.su (Postfix, from userid 1000)
            id 5089120210E4; Fri, 16 Sep 2016 13:28:53 +0300 (MSK)
        Message-Id: <20160916102853.5089120210E4@st15.os3.su>
        Date: Fri, 16 Sep 2016 13:28:53 +0300 (MSK)
        From: sudent15@st15.os3.su (Student15)
        Status: O
        Content-Length: 7
        Lines: 1

        test35
        \end{verbatim}


        \section{Installint mail client \\}
        I chose mutt - minimalistic mail client that can read and send mails. 
        \begin{enumerate}
            \item Sendmail stores mails in /var/mail/$username$ file and mutt reads them from that file.
            \item In format of ASCII text file, formed like mail in logs above.
        \end{enumerate}

        \section{Mail queue \\}
        \begin{enumerate}
            \item Sendmail queues mails in /var/spool/mqueue by itself.
            \item Like other queues it assures deleviry of every mail to recepient or return to sender.
            For example, if sender's mail server is off, mail will stay in queue and sendmail will try to send it later.
            \item /var/spool/mqueue
            \item with util name mailq and sendmail.
            For example we can watch all mails in queue by mail command and try to send them with sendmail -q.
        \end{enumerate}





\end{document}