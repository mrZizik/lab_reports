\documentclass[10pt]{article}
% Эта строка — комментарий, она не будет показана в выходном файле
\usepackage{ucs}
\usepackage[a4paper]{geometry}
\usepackage{a4wide}
\usepackage{enumitem}
\usepackage[utf8x]{inputenc } % Включаем поддержку UTF8
\title{SSN. Lab 1}
\date{19 October 2016}
\author{Ali Abdulmadzhidov}

\begin{document}
\renewcommand*\rmdefault{cmss}
  \maketitle
  \section{Crypto}
  \subsection{Overview}
  We knew about 2 types of ciphers: substituion and transposition. 
  Substitution - ciphers where units of plain text are changed by some rule to other.
  In transposition ciphers units stay unchaged, but chage their place in plaintext
  \subsection{Affine}
  Affine - monoalphabetic substitution cipher, where each letter changed to character by function \begin{equation} E(x) = (ax + b) mod m \end{equation}, where a and b are keys, x is numerical form of letter, counted from beginin of alphabet (A=0, b=1 ...) and m is numbers of symbols in that alphabet. Also a and m should be coprime, because if they are not, decryption might be impossible.
  Affine cipher has all weaknesses of monoalphabetic substitutional ciphers. There are only 12 numbers that are coprime with 26 and less then 26. Also only 26  values for b. 12*26=312 posible a,b combinations. Becouse of this, it can be easily carcked with bruteforce. Also it is weak for frequenty analize, and if cryptoanalyst can discover plaintext of 2 cipher characters, he can obtain key by solving simultaneous equation.
  For decryption we use function \begin{equation} D(x) = a^{-1}(x - b) mod m \end{equation}.

  \subsubsection{Encrypting example}
  \begin{enumerate}
  	\item Let plaint text be \begin{verbatim}`All world is stage'\end{verbatim} and a=9 b=3
  	\item Turn all symbols to numbers beginning from A (A=0, B=1 ... ) \\
  	plain text will be
  	 \begin{verbatim}0 11 11  22 14 17 11 3  8 18  18 19 0 6 4 \end{verbatim}
  	\item Symbol by symbol we process that numbers through 
  		\begin{equation} E(x) = (ax + b) mod m \end{equation}
  	For example. (9*0+3) mod 26 = 3. 
  	Result 
  		\begin{verbatim}3 24 24  19 25 0 24 4  23 9 9 18 3 5 13 \end{verbatim}
  	\item At the end we need to put symbols instead of numbers, inversing first step. 
  	Our ciphertext is - DYY TZAYE XJ JSDFN
  \end{enumerate}

  \subsubsection{Decrypting example}
  \begin{enumerate}
  	\item Let сiphertext be \begin{verbatim}DYY TZAYE XJ JSDFN\end{verbatim}. We also know key - a=9 b=3
  	\item Turn all symbols to numbers beginning from A (A=0, B=1 ... ) \\
  	plain text will be
  	 \begin{verbatim}3 24 24  19 25 0 24 4  23 9 9 18 3 5 13 \end{verbatim}
  	\item Next we need to compute \begin{equation}a^{-1}\end{equation} There're various methods to do that. At least we got that equal 3.
   	\item We produce computation by decryption formula for each number. For example \begin{equation} 3*(3-9) mod 26 = 0 \end{equation}
   	and get result \begin{verbatim}0 11 11 22 14 17 11 3 8 18 18 19 0 6 4\end{verbatim}

  	\item At the end we need to put symbols instead of numbers, inversing first step. Our plain text is - ALL WORLD IS STAGE 
  \end{enumerate}
  \subsection{Playfair}
  Playfair - simmetric and first litteral digram substitution cipher, invented by Charles Wheatstone, but named by Lord Playfair who promoted use of that cipher. Harder to break, that single letter substitutional ciphers, cause it's harder to make frequency analysis for bigrams. Nowadays it is not used cause even smartphones easily break it in seconds.

  \subsubsection{Encryption}
  To encrypt we use 5x5 square starting from keyword and finished by rest of ordered alphabet without J (suppose J==I) like:
  \begin{verbatim}  
  S O M E K
  Y A B C D
  F G H I L
  N P Q R T
  U V W X Z
  \end{verbatim}
  Also we divide our plaintext in bigrams like:
  \begin{verbatim}
  IN TE LX LI GE NC EX
  \end{verbatim}
  If there're repeated letters or last one hasn't got pair, we add X after letter.
  Then we use 4 rules to crypt that message with Playfair cipher.
  \begin{enumerate}
  	\item If 2 symbols of bigram stays on one row, we substitute them with right neighbors each.
  	\item .. stays on one col, subsitute with below letters.
  	\item .. stays on vertexes of rectangle, change them on letters, that stays on their row and also are vertexes of that rectangle
  	\item If there is no below or right side neighbor, we took first from opposite side (first row, or first col). 
  \end{enumerate}

  Let's crypt our message
  \begin{enumerate}
  	\item IN stays on vertexes of {F, I, X, U} rectangle and we change them to FR
  	\item TE change to RK
  	\item LX = IZ
  	\item LI is on one row, we change them to right neighbor, i->L, but L is last in row, we change it to first letter: L->F
  	\item GE = IO
  	\item NC = RY
  	\item EX = CE
  \end{enumerate}

  Decrypting goes same, but uses inversed rules.

  \subsection{ADFGVX}
  ADFGVX (ADFGX in origin) is cipher that was originally invented by Lt. Fritz Nebel and upgrageded in 1918. It uses both substitional and transpostitional methods, that made it very strong for usage in WWI. 
  Cipher is named after six characters that used in it. Those was chosen because they look very different in Morse code, that makes errors more rare.

  \susbsubsection{Encryption}
  Let's encrypt `RFC'.
  First we need 5x5 square filled with secret mixed alphabet - substition key


  \begin{verbatim}
  	A D F G X
  A Z X D A W
  D V R T U S
  F L M O Q E
  G Y I P F G
  X H K N B C
  \end{verbatim}

  I and J encrypted like one to make 26 six alphabet fit in 25 cells.

  Using that for each letter in plaintext we take row letter and column letter. \\
  R = DD F = GG C = XX
  DDGGXX \\ \\
  Second step is transposition.
  We write our ciphertext under transposition key.
  \begin{verbatim}
  H A L F
  D D G G
  X X
  \end{verbatim}
  Then we sort letter in HALF in alphabetical order, changing cols in table,
  \begin{verbatim}
  A F H L 
  D G D G 
  X   X
  \end{verbatim}
  and write by columns from up to below.
  We got ciphertext = DX GD XG

  Decryption is just inverse of each step using both keys.

  \subsubsection{Encrypting text with Vigenere}
  My plain text was 
  \begin{verbatim}
  	memories can be vile, repulsive little brutes. like children, i suppose.
  	haha. but can we live without them? alghough, why not? we aren't 
  	contractually tied down to rationality! there is no sanity clause! so when
  	you find yourself locked onto an unpleasant train of thought, heading for
  	the places in your past where the screaming is unbearable, remember
  	there's always madness. you can just step outside, and close the door
  	on all those dreadful things that happened. you can lock them away.
  	forever. madness is the emergency exit.
  \end{verbatim} \\
  Firstly I took out all punctuation and spaces \\
  \begin{verbatim}
  	memoriescanbevilerepulsivelittlebruteslikechildrenisupposehahabut
  	canwelivewithoutthemalghoughwhynotwearentcontractuallytieddowntor
  	ationalitythereisnosanityclausesowhenyoufindyourselflockedontoanu
  	npleasanttrainofthoughtheadingfortheplacesinyourpastwherethescrea
  	mingisunbearableremembertheresalwaysmadnessyoucanjuststepoutsidea
  	ndclosethedooronallthosedreadfulthingsthathappenedyoucanlockthema
  	wayforevermadnessistheemergencyexit
  \end{verbatim}
  Key=JOKER, after multiplying to length of ciphertext mod length of key, i have
  \begin{verbatim}
  	JOKERJOKERJOKERJOKERJOKERJOKERJOKERJOKERJOKERJOKERJOKERJOKERJOKERJO
  	KERJOKERJOKERJOKERJOKERJOKERJOKERJOKERJOKERJOKERJOKERJOKERJOKERJOKE
  	RJOKERJOKERJOKERJOKERJOKERJOKERJOKERJOKERJOKERJOKERJOKERJOKERJOKERJ
  	OKERJOKERJOKERJOKERJOKERJOKERJOKERJOKERJOKERJOKERJOKERJOKERJOKERJOK
  	ERJOKERJOKERJOKERJOKERJOKERJOKERJOKERJOKERJOKERJOKERJOKERJOKERJOKER
  	JOKERJOKERJOKERJOKERJOKERJOKERJOKERJOKERJOKERJOKERJOKERJOKERJOKERJO
  	KERJOKERJOKERJOKERJOKER
  \end{verbatim}

  Using Vigener's square I took column marked as encrypting letter, and row marked with letter that stays whos position is same as encrypting letter.

  \begin{verbatim}
  	vswsirscgrwpozzusbigdzcmmnzsxkuslvlcscpztsmlzurbiergetgxgolrqolykl
  	oxavuwfinrhrslchridjzqlfdurayhbyxnnobiecqyrkaomxljzvckrsnhffbdsijh
  	ssejzsxpcvovvrgxsjjbsxplzkyjngyaynbislowxhpxibwvutvsttsnsecckrl
  	wdvirboxxkaosrfohrslpvdlvjrsrxocbxyndvetngsrpxibtrbhglvasdlvbqbirv
  	wxkzbixfvjfkfcnfoqvvpovkqsbijjzgepbakhengccfdqkradgdwkndyykbwnirwr
  	mpfbsdlvmcyvfwovpkqcciuaskhwdzdlzwucxyjhregysxiuhcegrwzygbcvoqrfoi
  	jfasfiivonrvbgswkqsoqvauorthshmk
  \end{verbatim}

\subsubsection{Decrypting Vigener}
I got this ciphertext:
	\begin{verbatim}
	BUWCMKJILBMFMCWSNMQOJKXLBHWMUVAJKWZPICMMHPCSVLTLQZUCJXCHKZHJMVSA
	SNQLNWKJLRASFXUFRBXGVGMSLLEYEBNPGPWMHXQAIIIFGOJQSQGUTZBRQPFZLIEB
	EWEMOLKCJGQGDMKAQNQOFHDOJQGCAFHCEQWHZKTRQBSGSWYYCTMFMMNAEITGRVGR
	MYJUJEOCJWGCWSYVWJKLJBXVIZUZWWHLFUMFMYFGLGGHKCKWWSYVWVQFVBAYBPWM
	HXCHRTRQQFMOKLCXZVIMTVYWUENUELFGTVYOJCCZWIBPAVSHZMUMVKMGWABSEIPN
	ZWGYNRBCLLGLYQLRWENQSPJCXPEGOUYGXVQGKPXNIFYQWRVOIGYMZZTFWSPNYMAG
	AGTFQSHWIGIRWTWOHLANYMUCAGXCMVEYZADYPAXTSWECEIMGVTGCGO
	\end{verbatim}
Firstly i tried to find length of key.




\subsection{Encrypting with chosen cipher}
I chose Affien cipher and keys a=5 and b=7.
Plaintext
\begin{verbatim}
	There's a time when the operation of the machine becomes so odious, makes you so sick at heart, that you can't take part. You can't even passively take part, and you've got to put your bodies upon the gears and upon the wheels, upon the levers, upon all the apparatus, and you've got to make it stop. And you've got to indicate to the people who run it, to the people who own it, that unless you're free, the machine will be prevented from working at all
\end{verbatim}
After processing like in vigenere, we got
\begin{verbatim}
	theresatimewhentheoperationofthemachinebecomessoodiousmakesyousosickatheart
	thatyoucanttakepartyoucantevenpassivelytakepartandyouvegottoputyourbodiesup
	onthegearsanduponthewheelsupontheleversuponalltheapparatusandyouvegottomake
	itstopandyouvegottoindicatetothepeoplewhorunittothepeoplewhoownitthatunless
	yourefreethemachinewillbepreventedfromworkingatall
\end{verbatim}

Converted to numbers
\begin{verbatim}
	19 7 4 17 4 18 0 19 8 12 4 22 7 4 13 19 7 4 14 15 4 17 0 19 8 14 13 14 5 
	19 7 4 12 0 2 7 8 13 4 1 4 2 14 12 4 18 18 14 14 3 8 14 20 18 12 0 10 4 18 
	24 14 20 18 14 18 8 2 10 0 19 7 4 0 17 19 19 7 0 19 24 14 20 2 0 13 19 19 
	0 10 4 15 0 17 19 24 14 20 2 0 13 19 4 21 4 13 15 0 18 18 8 21 4 11 24 19 
	0 10 4 15 0 17 19 0 13 3 24 14 20 21 4 6 14 19 19 14 15 20 19 24 14 20 17 
	1 14 3 8 4 18 20 15 14 13 19 7 4 6 4 0 17 18 0 13 3 20 15 14 13 19 7 4 22 
	7 4 4 11 18 20 15 14 13 19 7 4 11 4 21 4 17 18 20 15 14 13 0 11 11 19 7 4 
	0 15 15 0 17 0 19 20 18 0 13 3 24 14 20 21 4 6 14 19 19 14 12 0 10 4 8 19 
	18 19 14 15 0 13 3 24 14 20 21 4 6 14 19 19 14 8 13 3 8 2 0 19 4 19 14 19 
	7 4 15 4 14 15 11 4 22 7 14 17 20 13 8 19 19 14 19 7 4 15 4 14 15 11 4 22 
	7 14 14 22 13 8 19 19 7 0 19 20 13 11 4 18 18 24 14 20 17 4 5 17 4 4 19 7 
	4 12 0 2 7 8 13 4 22 8 11 11 1 4 15 17 4 21 4 13 19 4 3 5 17 14 12 22 14 
	17 10 8 13 6 0 19 0 11 11
\end{verbatim}

Each number we process through affine function \begin{equation} E(x) = (ax + b) mod m \end{equation}

\begin{verbatim}
	24 16 1 14 1 19 7 24 21 15 1 13 16 1 20 24 16 1 25 4 1 14 7 24 21 25 20 25 
	6 24 16 1 15 7 17 16 21 20 1 12 1 17 25 15 1 19 19 25 25 22 21 25 3 19 15 
	7 5 1 19 23 25 3 19 25 19 21 17 5 7 24 16 1 7 14 24 24 16 7 24 23 25 3 17 
	7 20 24 24 7 5 1 4 7 14 24 23 25 3 17 7 20 24 1 8 1 20 4 7 19 19 21 8 1 10 
	23 24 7 5 1 4 7 14 24 7 20 22 23 25 3 8 1 11 25 24 24 25 4 3 24 23 25 3 14 
	12 25 22 21 1 19 3 4 25 20 24 16 1 11 1 7 14 19 7 20 22 3 4 25 20 24 16 1 
	13 16 1 1 10 19 3 4 25 20 24 16 1 10 1 8 1 14 19 3 4 25 20 7 10 10 24 16 1 
	7 4 4 7 14 7 24 3 19 7 20 22 23 25 3 8 1 11 25 24 24 25 15 7 5 1 21 24 19 
	24 25 4 7 20 22 23 25 3 8 1 11 25 24 24 25 21 20 22 21 17 7 24 1 24 25 24 
	16 1 4 1 25 4 10 1 13 16 25 14 3 20 21 24 24 25 24 16 1 4 1 25 4 10 1 13 
	16 25 25 13 20 21 24 24 16 7 24 3 20 10 1 19 19 23 25 3 14 1 6 14 1 1 24 
	16 1 15 7 17 16 21 20 1 13 21 10 10 12 1 4 14 1 8 1 20 24 1 22 6 14 25 15 
	13 25 14 5 21 20 11 7 24 7 10 10 
\end{verbatim}

And convert to characters like we done earlier

\begin{verbatim}
	yqbobthyvpbnqbuyqbzebohyvzuzgyqbphrqvubmbrzpbttzzwvzdtphfbtxzdtztvrfhyqbhoy
	yqhyxzdrhuyyhfbehoyxzdrhuybibuehttvibkxyhfbehoyhuwxzdiblzyyzedyxzdomzwvbtde
	zuyqblbhothuwdezuyqbnqbbktdezuyqbkbibotdezuhkkyqbheehohydthuwxzdiblzyyzphfb
	vytyzehuwxzdiblzyyzvuwvrhybyzyqbebzekbnqzoduvyyzyqbebzekbnqzznuvyyqhydukbtt
	xzdobgobbyqbphrqvubnvkkmbeobibuybwgozpnzofvulhyhkk
\end{verbatim}



\end{document}