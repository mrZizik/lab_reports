\documentclass[10pt]{article}
% Эта строка — комментарий, она не будет показана в выходном файле
\usepackage{ucs}
\usepackage[a4paper]{geometry}
\usepackage{a4wide}
\usepackage{enumitem}
\usepackage[utf8x]{inputenc } % Включаем поддержку UTF8
\title{Essential Skills. Lab 2}
\date{09 September 2016}
\author{Ali Abdulmadzhidov}

\begin{document}
\renewcommand*\rmdefault{cmss}
  \maketitle
  \section{Debian packagins system    \newline}

  \begin{enumerate}
    \item How does it work?
        Debian package system has various different level tools for user. The base of package system is dpkg - Debian package manager. In spite of man page says that the description of steps in installing or removing process is inadequate, i'll try.
        
        \subsubsection*{*.deb - Debian package}
            Debian package contains three sections:
                \begin{description}
                    \item[Global header] Contains version of debian-binary package. Current is 2.0
                    \item[Control section] Meta information about package. Confiles, preinst, postinst, prerm, postrm scripts, list of dependecies, md5 hashsumm, brief description. 
                    \item[Data section] Includes installable files
                \end{description}
        \subsubsection*{Installation}
        dpkg -i
        \begin{enumerate}[label=\arabic*,ref=\theenumi]
            \item Extract of control files of the package
            \item If another version of package is already installed, execute prerm script of the outdated package
            \item If package has preinst script, execute it.
            \item Run postrm script of old package
            \item Unpack new, and at the same time backup old files.
            \item Configure package
        \end{enumerate}

        \subsubsection*{Configuration}
        dpkg -configure
        \begin{enumerate}[label=\arabic*,ref=\theenumi]
            \item Unpack conffiles, and backup old conffiles.
            \item Run postinit script
        \end{enumerate}

        \subsubsection*{Removing}
        dpkg -remove
        Removes an installed package, but lefts connfiles
        \begin{enumerate}[label=\arabic*,ref=\theenumi]
            \item Run prerm script
            \item Remove isntalled files
            \item Run posterm script
        \end{enumerate}

        \subsubsection*{Purge}
        dpkg -purge
        Purges (removes) an installed or removed package with connfiles.
        \begin{enumerate}[label=\arabic*,ref=\theenumi]
            \item Remove package
            \item Run posterm script
        \end{enumerate}




    \begin{verbatim}
    wget ftp://ftp.sendmail.org/pub/sendmail/sendmail.8.15.2.tar.gz
    \end{verbatim}
    \item Cheking signature of file
    \begin{verbatim}
    wget ftp://ftp.sendmail.org/pub/sendmail/sendmail.8.15.2.tar.gz.sig
    wget http://www.sendmail.com/sm/open_source/security/pgp_keys/sendmail2015.asc
    gpg --import sendmail2015.asc
    gpg --verify sendmail.8.15.2.tar.gz.sig
    \end{verbatim}
    \item Extract souces
    \begin{verbatim}
    tar -xzvf sendmail.8.15.2.tar.gz
    \end{verbatim}
    \item Read readme file first... No, really for the first time I just builded it and can't start. Then remove all staff and read readme file. 
    \begin{verbatim}
    cat README
    \end{verbatim}
    \item Add 
    \begin{verbatim}
    sudo apt install libapache2-mod-php 
    \end{verbatim}
    \item Restarting apache2. 
    \begin{verbatim}
    sudo service apache2 restart
    \end{verbatim}
    \item Go to http://localhost/dokuwiki/doku.php and add our first page about Federative Wiki
  \end{enumerate}

  \section{XML entities creating    \newline}
  \begin{enumerate}
    \item XML + DDT
    \begin{verbatim}
        <!DOCTYPE root
        [
        <!ELEMENT root (animal)>
        <!ELEMENT animal (name,skin,voice)>
        <!ELEMENT name (#PCDATA)>
        <!ELEMENT skin (#PCDATA)>
        <!ELEMENT voice (#PCDATA)>
        <!ENTITY  skin "have skin">
        <!ENTITY  fur "have fur">
        <!ENTITY  feathers "have feathers">
        <!ENTITY  bark "bark">
        <!ENTITY  mur "mur">
        <!ENTITY  bee "bee">
        <!ENTITY  brr "brr">
        <!ENTITY  mooo "mooo">
        <!ENTITY  oink "oink">
        <!ENTITY  whistle "whistle">
        <!ENTITY  no_voice "no_voice">
        ]>
        <root>
<animal>
            <name>cow</name>
            <skin>&skin;</skin>
            <voice>&mooo;</voice>
        </animal>
        <animal>
            <name>sheep</name>
            <skin>&fur;</skin>
            <voice>&bee;</voice>
        </animal>
        <animal>
            <name>horse</name>
            <skin>&skin;</skin>
            <voice>&brr;</voice>
        </animal>
        <animal>
            <name>pig</name>
            <skin>&skin;</skin>
            <voice>&oink;</voice>
        </animal>
        <animal>
            <name>mockingbird</name>
            <skin>&feathers;</skin>
            <voice>&whistle;</voice>
        </animal>
        <animal>
            <name>eal</name>
            <skin>&feathers;</skin>
            <voice>&whistle;</voice>
        </animal>
</root>
    \end{verbatim}
    \item XS SCHEME
    \begin{verbatim}
    <?xml version="1.0" encoding="utf-8"?>
    <xs:schema xmlns:xs="http://www.w3.org/2001/XMLSchema">
    <xs:element name="root">
        <xs:complexType>
        <xs:sequence>
            <xs:element name="animal" maxOccurs="unbounded">
                <xs:complexType>
                <xs:sequence>
                    <xs:element name="name" type="xs:string"/>
                    <xs:element name="voice" type="xs:string"/>
                    <xs:element name="skin" type="xs:string"/>
                </xs:sequence>
                </xs:complexType>
            </xs:element>
        <xs:sequence>
        </xs:complexType>
      </xs:element>
    </xs:schema>
    \end{verbatim}
  \end{enumerate}
\section*{XSLT and CSS \\}
\begin{enumerate}
\item XSLT
\begin{verbatim}
<?xml version='1.0' encoding='UTF-8'?>
<xsl:stylesheet version='1.0' xmlns:xsl='http://www.w3.org/1999/XSL/Transform'>
<xsl:template match='/'>
  <html>
  <body>
  <h2>Animals</h2>
    <table border='1'>
      <tr>
        <th style='text-align:left'>Name</th>
        <th style='text-align:left'>Skin</th>
        <th style='text-align:left'>Voice</th>
      </tr>
      <xsl:for-each select='root/animal'>
      <tr>
        <td class='name-vl'><xsl:value-of select='name'/></td>
        <td class='skin-vl'><xsl:value-of select='skin'/></td>
        <td class='voice-vl'><xsl:value-of select='voice'/></td>
      </tr>
      </xsl:for-each>
    </table>
  </body>
  </html>
</xsl:template>
</xsl:stylesheet>
\end{verbatim}

\item CSS
\begin{verbatim}
td {
    background: #eeccbb;
}
th {
    background: #ffff00;  
}
.name-vl {
    color: #0f0;
}
.name-vl:hover {
    color: #00f;
}
.voice-vl {
    color: #00f;
}
.voice-vl:hover {
    color: #0f0;
}
.skin-vl {
    color: #f00;
}
.skin-vl:hover {
    color: #f0f;
}
\end{verbatim}
\end{enumerate} 
\end{document}